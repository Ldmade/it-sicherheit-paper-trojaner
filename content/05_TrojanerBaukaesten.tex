\section{Trojaner Baukästen}\label{sec:trojanerBaukaesten}
Um einen Trojaner zu entwickeln braucht man Kenntnisse über Betriebssysteme, Netzwerke und Programmiersprachen.
Das heißt um Geld durch einen Trojaner zu verdienen, benötigt man erheblichen Aufwand.\\
Aus diesem Grund werden sogenannte Trojaner Baukästen im Internet angeboten. Somit können Kriminelle mit wenig Computerwissen, Schaden anrichten und sich bereichern. Dabei sind die Trojaner Baukasten oftmals leicht zu bedienen und haben viele nützliche Funktionen. Somit lässt sich innerhalb weniger Minuten ein Trojaner erstellen und konfigurieren. Wie einfach es geht, zeigt Beispielsweise das Youtube Tutorial in \cite{YOUTUBE}, welches die Anwendung von DarkComet zeigt.\\
Grundsätzlich bieten ein Trojanen Baukasten an, einen Trojaner zu erstellen und als Listener zu fungieren. Der Listener wartet so lange, bis der Trojaner sich meldet und erlaubt dem Angreifer funktionen auszuführen.\\
Der Angreifer kann seinen Trojaner beliebig konfigurieren. Folgend sind einige Funktionen beschrieben.\cite{PANDA}
\begin{description}
\item[Return mode] Wie soll der Trojaner mit dem Angreifer kommunizieren und die gestohlenen Daten übertragen? HTTPS, SMTP, ...?
\item[Passwörter] Welche Arten von Passwörter sollen gestohlen werden? E-Mail, Windows, ...?
\item[Keylogger] Soll der Trojaner die Tastatureingaben mitschneiden?
\item[Run Methode] Wie soll der Trojaner ausgeführt werden? Soll er sich in den Autostart setzen? Soll er als Service ausgeführt werden?
\item[Backdoor] Soll der Trojaner Ports öffnen, mit denem man sich auf dem Opfer anmelden kann?
\item[Kill] Sollen spezielle Services deaktiviert werden? Hier bieten viele Baukästen auch eine Liste von bekannten Services von Antiviren Programme.
\item[Screenshots] Soll der Trojaner Screenshots des Opfers machen?
\item[Wurm] Soll sich der Trojaner selbst verbreiten?
\end{description}
Nach der Konfiguration kann der Trojaner kompiliert und erzeugt werden. Ein Trojaner Baukasten bietet meist zusätzlich die Möglichkeit die .exe Date zu verschleiern, zum Beispiel mit einem angegeben Icon oder einer falschen Endung (z.B. .mp3). Auch werden funktionen angeboten um den Trojaner zu verschleiern, damit das Signatur basierte verfahren von Antivren Software keinen Erfolg hat.\\
Der einzige Nachteil eines Baukasten ist, das er nur eine bestimmte Art von Trojaner erzeugen kann. Dieser ist zwar unterschiedlich konfiguriert, ist jedoch immer Ähnlich. Aus diesem Grund können Antiviren Softwarehersteller, durch analysieren des Baukasten die eigene Erkennung verbessern und somit die Gefahr einen bestimmten Baukasten minimieren. Wurde ein Baukaste auf diese Weise analyisiert, ist er für Kriminelle weniger Wert.\cite{FOCUS}

% Sub7, MPack, Pinch 3, Power Grabber, Bifrost, DarkComet