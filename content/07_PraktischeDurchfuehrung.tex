\section{Praktische Durchführung}\label{sec:praktischeDurchfuehrung}
Im folgenden Kapitel wird das bereits vorgestelle Metasploit
Framework genutzt um einen beispielhaften Trojaner zu erstellen.

\subsection{Aufbau}\label{sec:praktischeDurchfuehrung-aufbau}
Für einen möglichst einfachen und gleichzeitig verhältnismäßig sicheren Ablauf
wird das Labor in einer virtualisierten Umgebung ohne Verbindung zum Internet durchgeführt.
Beispielsweise VirtualBox bietet hierfür eine \glqq Internal Networking\grqq{} Einstellung, unter welcher lediglich die virtuellen Maschinen untereinander kommunizieren können.
Ist beispielsweise der Angreifer ein realer Computer und dient gleichzeitig dem Opfer als Host, so kann durch ein \glqq Host-ony networking\grqq{} sichergestellt werden, dass zumindest der Gast keinen Zugriff zum Internet hat.

\begin{figure}[h!]
	\centering
	\begin{tikzpicture}[,>={Stealth[round]},shorten >=1pt,auto,semithick]
		\tikzstyle{pc}=[rectangle,draw=black,minimum size=30pt,inner sep=5pt,align=center]

		\node[pc]      (A) at (1,0)  {Angreifer \\ (Kali Linux, Metasploit)};
		\node[pc]      (V) at (5,0)  {Opfer\\(Windows 10)};

		\path (A) edge (V) ;
	\end{tikzpicture}
	\caption{Übersicht} \label{fig:uebersicht}
\end{figure}

An der praktischen Durchführung sind, wie in Abbildung \ref{fig:uebersicht} dargestellt, zwei Computer beteiligt.
Auf dem Angreifer läuft das Betriebssystem \glqq Kali Linux\grqq{}, welches bereits das Metasploit Framework enthält.
Außerdem wird der späteren Trojaner über einen Apache Webserver an das Opfer zu übergeben.
Das Opfer nutzt das weit verbreitete Betriebssystem Windows 10.