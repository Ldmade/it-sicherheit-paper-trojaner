\section{Antivirus-Software}\label{sec:antivirus}
Zum Schutz vor Trojaner und Viren hilft Antivirus-Software.
Hauptaufgabe dieser Software ist es Malaware zu entfernen oder in Quarantäne zu verschieben.
Bevor die Malware entfernen werden kann, muss die Software diese entdecken.
Dabei haben sich im Laufe der Zeit mehrere Techniken entwickelt, die man grundsätzlich 
in zwei Arten unterteilen kann.\cite{ANTNE1}
\begin{itemize}
    \item Signaturbasierte Erkennung
    \item Heuristischen Erkennung
\end{itemize}
Die \textbf{signaturbasierte Erkennung} scannt Dateien auf dem zu schützenden Computer 
und bildet deren Signaturen. 
Diese Signaturen werden mit einer Datenbank bekannter Malwaresignaturen abgeglichen.
Um auch neuere Viren und Trojaner zu erkennen müssen Anbieter von Antiviren Software diese Datenbanken regelmäßig aktualisieren.
Die signaturbasierte Erkennung kann erschwert werden, indem die eigentliche Signatur der Malware durch Verschlüsselung verschleiert wird.
Aus diesem Grund werden weitere Algorithmen angewandt, welche die verschlüsselten Formen wieder auf einfache Formen zurückwandeln.
Durch diese Transformationen kann man viele Trojaner bereits durch wenige Signaturen erkennen \cite{ANTNE2}.

Der Schwachpunkt der signaturbasierten Erkennung ist, dass meist nur bereits bekannte Malware aufgedeckt werden kann,
da von neuer Malware die Signatur noch nicht bekannt ist.
Deshalb wird zusätzlich auch die \textbf{heuristische Erkennung} genutzt. 
Diese analysiert das Verhalten einer Software und gleicht es mit einer Datenbank ab.
Ist das Verhalten ähnlich zu einer Malware, wird die Software als Malware erkannt und beispielsweise der Zugriff der Software blockiert.
Für die heuristische Erkennung gibt es unterschiedliche Ansätze
Einerseits kann der normale Betrieb einer Software analysiert werden,
oder andererseits die Software in einer isolierten Umgebung ausgeführt werden um festzustellen,
ob es sich um Malware handelt. 
Letzteres ist jedoch aufwändig, da eine virtuelle Umgebung geschaffen werden muss \cite{ANTNE3}.

