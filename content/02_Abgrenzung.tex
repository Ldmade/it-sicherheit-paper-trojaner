\section{Begriffserklärung: Virus, Trojaner, Wurm}\label{sec:abgrenzung}
Viren, Trojaner und Würmer sind verschiedene Umsetzungen von Schadsoftware,
verfolgen allerdings ähnliche Ziele. Daten sind das wichtigste Gut der
Informationstechnologie. Die meisten Schadprogramme zielen auf diese ab, sei es
das Ziel die Daten zu löschen, zu blockieren, zu modifizieren oder zu kopieren.
Hat ein Hacker Zugriff auf die Daten, so können weitere Aktionen folgen.
Zum Beispiel kann Lösegeld für die Daten gefordert werden.
Ein anderes Ziel einiger Schadprogramme ist es die System- und/oder Netzwerkleistung
einzuschränken. Auch hier können darauf basierend weitere Schritte folgen.
Doch was ist jetzt der Unterschied zwischen 'Viren', 'Würmern' und 'Trojanern'?
Um diese Frage zu klären, werden im Folgenden die verschiedenen Merkmale dargestellt.
\cite{KASTRO}\\

\textbf{Was ist ein Virus?}
\begin{itemize}
    \item \textbf{Verbreitung: } 
        \begin{itemize}
            \item repliziert sich selbst
            \item verbreitet sich, in dem das Wirtsprogramm verbreitet wird
        \end{itemize}
    \item \textbf{Funktionsweise: } 
        \begin{itemize}
            \item hängt an einem Wirtsprogramm
            \item schläft bis das Wirtsprogramm ausgeführt wird
        \end{itemize}
\end{itemize}

\textbf{Was ist ein Wurm?}
\begin{itemize}
    \item wie ein Virus repliziert sich ein Wurm selbst
    \item benötigen nach 'infektion' eines Rechners kein Wirtsprogramm mehr
    \item verbreiten sich sehr leicht und schnell
    \item bedrohen so nicht nur einzelne Rechner, sondern ganze Netzwerke
\end{itemize}

\textbf{Was ist ein Trojaner?}
\begin{itemize}
    \item versteckte Zusatzfunktionalität, die sich in 'nützlichem' Programm versteckt
    \item repliziert sich nicht selbstständig
    \item ruft das Opfer das 'nützliche' Programm auf, werEden schädliche Routinen ausgeführt
\end{itemize}
\cite{ABGVT}