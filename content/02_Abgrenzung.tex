\section{Begriffserklärung: Virus, Trojaner, Wurm}\label{sec:abgrenzung}
Viren, Trojaner und Würmer bezeichnen verschiedene Varianten von Schadsoftware,
verfolgen allerdings ähnliche Ziele.
Die meisten Schadprogramme haben das Ziel Daten zu löschen, zu blockieren, zu modifizieren oder zu kopieren.
Zum Beispiel kann Lösegeld für die Freigabe der Daten gefordert werden.
Ein weiteres Ziel einiger Schadprogramme ist es, die System- und/oder Netzwerkleistung
einzuschränken. Auch hier können darauf basierend weitere Schritte folgen.
Doch was ist der Unterschied zwischen \textit{Viren}, \textit{Würmern} und \textit{Trojanern}?
Um diese Frage zu klären, werden im Folgenden die verschiedenen Merkmale dargestellt.
\cite{KASTRO}\\

\textbf{Was ist ein Virus?}
Wie in der Medizin benötigt der Virus einen Wirt bzw. ein Wirtsprogramm an das 
er sich anhängen kann. Ein Virus kopiert sich selbst in verschiedene 
Wirts-Dateien. Er ist hierbei allerdings passiv und wartet auf eine Aktion des Opfers. 
Dies kann zum Beispiel das Ausführen des Wirtsprogramms sein. Auch bei der Verbreitung in ein anderes
System ist ein Virus passiv. Ein weiteres System wird nur dann infiziert, wenn
eine infizierte Wirts-Datei durch einen Benutzer übertragen wird. \\

\textbf{Was ist ein Wurm?}
Im Gegensatz zu einem Virus ist der Wurm ein aktives Schadprogramm. Es kann sich
selbst replizieren und in weiteren Systeme verbreiten. Am 25. Januar 2003 schaffte es 
der Wurm \textit{SQL Slammer} innerhalb von 30 Minuten 75.000 Opfer zu infizieren. Durch
eine kleine Größe können sich Würmer sehr leicht und schnell verbreiten. Sie
bedrohen so nicht nur einzelne Rechner, sondern ganze Netzwerke. Die Verbreitung
funktioniert meist über E-Mail- oder Chat-Programme, die ferngesteuert werden.\\
% \cite[SLAM] - https://de.wikipedia.org/wiki/SQL_Slammer

\textbf{Was ist ein Trojaner?}
Ein Trojanisches Pferd im Allgemeinen ist ein harmlos aussehendes Objekt, das von
einem Angreifer als Tarnung verwendet wird. Dies ist auch das Funktionsprinzip eines \textit{Trojaners}. 
Ein Opfer lädt beispielsweise ein vermeintlich nützliches Programm herunter.
Klassischerweise sind hier YouTube-Downloader zu nennen.
Neben der erwarteten Funktionalität enthält
das Programm beispielsweise eine versteckte Funktionalität die eine Verbindung zu dem Rechner
des Angreifers aufzubauen. Ein Trojaner verbreitet wie ein Virus sich nicht selbst, ein Benutzer 
muss aktiv werden und ein infiziertes Programm herunterladen und aufrufen, damit die
schädlichen Routinen des Trojaners ausgeführt werden.
\cite{ABGVT}