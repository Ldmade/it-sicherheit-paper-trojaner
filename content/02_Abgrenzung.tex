\section{Begriffserklärung: Virus, Trojaner, Wurm}\label{sec:abgrenzung}
Alle Schadprogramme können die gleichen Ziele verfolgen. Diese sind allgemein bezeichnet: 
\begin{itemize}
    \item Daten löschen
    \item Daten blockieren
    \item Daten modifizieren
    \item Daten kopieren
    \item System-/Netzwerkleistung einschränken
\end{itemize} 
Doch verschiedene Schadprogramme haben unterschiedliche Merkmale und werden anhand dieser in Arten eingeteilt. 
Drei dieser Arten sind 'Virus', 'Wurm' und 'Trojaner'. Um diese von einander zu unterscheiden sind im folgenden die wesentlichen Merkmale dargestellt.
\cite{KASTRO}
\break
\textbf{Was ist ein Virus?}
\begin{itemize}
    \item repliziert sich selbst
    \item hängt an einem Wirtsprogramm
    \item schläft oft, bis das Wirtsprogramm ausgeführt wird
    \item verbreitet sich, in dem das Wirtsprogramm verbreitet wird
\end{itemize}
\textbf{Was ist ein Trojaner?}
\begin{itemize}
    \item versteckte Zusatzfunktionalität, die sich in 'nützlichem' Programm versteckt
    \item repliziert sich nicht selbstständig
    \item ruft das Opfer das 'nützliche' Programm auf, werEden schädliche Routinen ausgeführt
\end{itemize}
\textbf{Was ist ein Wurm?}
\begin{itemize}
    \item benötigen nach 'infektion' eines Rechners kein Wirtsprogramm mehr
    \item verbreiten sich sehr leicht und schnell
    \item wie ein Virus repliziert sich ein Wurm selbst
    \item bedrohen so nicht nur einzelne Rechner, sondern ganze Netzwerke
\end{itemize}
\cite{ABGVT}