\section{Bekannte Trojaner}\label{sec:bekannteTrojaner}
Trojaner werden anhand ihrer Art voneinander abgegrenzt. Die Art spiegelt das Ziel wieder, das ein Trojaner verfolgt. 
\begin{itemize}
    \item  \textbf{Backdoor}: Öffnet eine Hintertür um einen Computer oder ganzes Netzwerk von außen zu steuern bzw. zu administrieren
    \item \textbf{Clicker}: Bringt ein Opfer dazu, bestimmte Webseiten (Werbung oder kostenpflichtig) aufzurufen 
    \item \textbf{Dropper}: Installiert weitere schädliche Software und versteckt diese
    \item \textbf{Linker}: Verbindet ein schädliches Programm mit einem vom Nutzer gewünschten Programm. Ziel ist es, das schädliche Programm im gewünschten Programm zu verstecken.
    \item \textbf{Downloader}: Lädt versteckte Software herunter (von SPAM bis hin zu Spy-Programmen)
    \item \textbf{Spy}: Spioniert den Rechner aus um Benutzerdaten, Screenshots, Tastatureingaben, Festplatteninhalte etc. zu erhalten.
\end{itemize}
Die verschiedenen Arten von Trojanern werden oftmals kombiniert. So können daraus mächtige Werkzeuge werden. 
Die 3 bekanntesten Trojaner, die außerdem als Grundlage für viele weitere Trojaner dienten, werden im Folgenden kurz vorgestellt. 

\subsection{Zbot-Trojaner}
Der Zbot-Trojaner ist ein Trojaner der Zeus Bot-Familie. Er war einer der ersten
Trojaner die als Rahmenwerk verkauft wurde, das heißt als Malware mit einem Konfigurations-
Tool zur Individualisierung. Ziel des Zbot-Trojaners ist es möglichst viele Informationen
des Benutzers zu stehlen um im Anschluss durch Erpressung vom Anwender Geld zu 
erzielen. Die Informationen werden beispielsweise durch Protokollierung der Tastatureingaben erlangt. 
Es könnte sich hier zum Beispiel um Anmeldedaten von E-Mail-Konten, Sozialen Netzwerken oder Online-Banking
handeln.
Der Zeus-Trojaner ist von Anti-Virusprogrammen sehr schwer zu entdecken, da dieser 'stealth techniques' verwendet. 
Typischerweise versteckt sich der Trojaner im Speicher. Dazu wird der original Speicher kopiert. Anfragen von anderen Programmen werden auf die Kopie weitergeleitet.
So kann der Trojaner unbemerkt den Speicher verändern.
Eine Abwandlung eines Zeus-Trojaners ist zum Beispiel der \textbf{Reveton-Trojaner}.
Dieser setzt den klassischen Ransomware Ansatz (Verschlüsselung der Opfer-Daten) um und ist bekannter unter dem Namen 'BKA-Trojaner'.
Ziel ist es den Zugriff auf das System zu blockieren und Lösegeld für die Freigabe
zu fordern. Der Reveton-Trojaner verwendet offizielle Logos um gegenüber der Opfer 
seriös zu wirken.

\subsection{Vundo-Trojaner}
Der Vundo-Trojaner verfolgt das Ziel Werbung für gefährliche bzw. gefälschte Antivirus-
Programme anzuzeigen und so Drive-By-Downloads zu starten. Der Trojaner installiert
meist eine Kombination aus 'Browser Helper Objects' und 'DLLs'. Um unerkannt zu bleiben
deaktiviert er außerdem Firewalls, Antiviren-Programme, Windows Updates, etc. 
\cite{BEKTRO}