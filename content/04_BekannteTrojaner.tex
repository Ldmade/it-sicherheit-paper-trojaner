\section{Bekannte Trojaner}\label{sec:bekannteTrojaner}
Trojaner werden anhand ihrer Art voneinander abgegrenzt. Die Art spiegelt das Ziel wieder, das ein Trojaner verfolgt. 
\begin{itemize}
    \item  \textbf{Backdoor}: Öffnet eine Hintertür um einen Computer oder ganzes Netzwerk von außen zu steuern bzw. zu administrieren
    \item \textbf{Clicker}: Bringt ein Opfer dazu, bestimmte Webseiten (Werbung oder kostenpflichtig) aufzurufen 
    \item \textbf{Dropper}: Installiert weitere schädliche Software und versteckt diese
    \item \textbf{Linker}: Verbindet ein schädliches Programm mit einem vom Nutzer gewünschten Programm. Ziel ist es, das schädliche Programm im gewünschten Programm zu verstecken.
    \item \textbf{Downloader}: Lädt versteckte Software herunter (von SPAM bis hin zu Spy-Programmen)
    \item \textbf{Spy}: Spioniert den Rechner aus um Benutzerdaten, Screenshots, Tastatureingaben, Festplatteninhalte etc. zu erhalten.
\end{itemize}
Die verschiedenen Arten von Trojanern werden oftmals kombiniert. So können daraus mächtige Trojaner werden. 
Die 3 bekanntesten Trojaner, die außerdem als Grundlage für viele weitere Trojaner dienten, werden im Folgenden kurz vorgestellt. 
\subsection{Zbot-Trojaner}
\begin{itemize}
    \item Trojaner der Zeus Bot-Familie
    \item einer der ersten Trojaner die als Rahmenwerk verkauft wurden
    \item Rahmenwerk = Malware mit Konfigurations-Tool zur Individualisierung
    \item Stiehlt durch z.B. Protokollierung der Tastatureingaben Zugangsdaten von Email-Konten, Sozialen Netzwerken, Online-Banking usw.
    \item Hauptziel: Durch Erpressung oder gestohlene Zugangsdaten Feld vom Anwender erlangen.
\end{itemize}
Der Zeus-Trojaner ist von Anti-Virusprogrammen sehr schwer zu entdecken, da dieser 'stealth techniques' verwendet. 
Typischerweise versteckt sich der Trojaner im Speicher. Dazu wird der original Speicher kopiert. Anfragen von anderen Programmen werden auf die Kopie weitergeleitet.
So kann der Trojaner unbemerkt den Speicher verändern.

\subsection{Vundo-Trojaner}
\begin{itemize}
    \item Werbung für gefährliche und gefälschte Antivirus-Programmen
    \item deaktiviert Firewalls, Antiviren-Programme, Windows Updates etc.
    \item Installiert meist eine Kombination aus 'Browser Helper Object' und 'DLL'
    \item Ziel ist es Werbung im Browser anzuzeigen und Drive-By-Downloads zu starten 
\end{itemize}
\subsection{Reveton-Trojaner}
\begin{itemize}
    \item Ransomware
    \item basiert auf dem Zeus-Trojaner
    \item als BKA-Trojaner bekanntesten
    \item Zugang zum Computer wird blockiert
    \item Opfer soll über anonyme Prepaid-Systeme Lösegeld bezahlen
    \item es werden offizielle Logos verwendet um seriös zu wirken
\end{itemize}
\cite{BEKTRO}