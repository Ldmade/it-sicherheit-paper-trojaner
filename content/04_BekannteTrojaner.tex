\section{Bekannte Trojaner}\label{sec:bekannteTrojaner}
Trojaner werden anhand ihrer Art voneinander abgegrenzt. Die Art spiegelt das Ziel wieder, das ein Trojaner verfolgt. 
\begin{itemize}
    \item Backdoor: Öffnet eine Hintertür um einen Computer oder ganzes Netzwerk von außen zu steuern bzw. zu administrieren
    \item Clicker: Bringt ein Opfer dazu, bestimmte Webseiten (Werbung oder kostenpflichtig) aufzurufen 
    \item Dropper: Installiert weitere schädliche Software und versteckt diese
    \item Linker: Verbindet ein schädliches Programm mit einem vom Nutzer gewünschten Programm. Ziel ist es, das schädliche Programm im gewünschten Programm zu verstecken.
    \item Downloader: Lädt versteckte Software herunter (von SPAM bis hin zu Spy-Programmen)
    \item Spy: Spioniert den Rechner aus um Benutzerdaten, Screenshots, Tastatureingaben, Festplatteninhalte etc. zu erhalten.
\end{itemize}
Die verschiedenen Arten von Trojanern werden oftmals kombiniert. So können daraus mächtige Trojaner werden. 
Die 3 bekanntesten Trojaner, die außerdem als Grundlage für viele weitere Trojaner dienten werden im Folgenden kurz vorgestellt. 
\begin{itemize}
    \item Zbot-Trojaner 
    \begin{itemize}
        \item Trojaner der Zeus Bot-Familie
        \item einer der ersten Trojaner die als Rahmenwerk verkauft wurden
        \item Rahmenwerk = Malware mit Konfigurations-Tool zur Individualisierung
        \item Stiehlt durch z.B. Protokollierung der Tastatureingaben Zugangsdaten von Email-Konten, Sozialen Netzwerken, Online-Banking usw.
        \item Hauptziel: Durch Erpressung oder gestohlene Zugangsdaten Feld vom Anwender erlangen.
    \end{itemize}
    \item Vundo-Trojaner
    \item Reveton-Trojaner
\end{itemize}