\section{Trojaner}\label{sec:trojaner}
Trojaner werden anhand ihrer Art unterschieden.
Diese spiegelt das Ziel wieder, welches ein Trojaner verfolgt.
Hierzu zählen:
\begin{description}
    \item[Backdoor] Öffnet eine Hintertür um einen Computer oder ein ganzes Netzwerk von außen zu steuern bzw. zu administrieren
    \item[Clicker] Bringt ein Opfer dazu, bestimmte Webseiten (z.B. Werbung) aufzurufen 
    \item[Dropper] Installiert weitere (ggf. schädliche) Software und versteckt diese
    \item[Linker] Verbindet ein schädliches Programm mit einem vom Nutzer gewünschten Programm. Ziel ist es, das schädliche Programm im gewünschten Programm zu verstecken
    \item[Downloader] Lädt versteckte Software herunter (von SPAM bis hin zu Spy-Programmen)
    \item[Spy] Spioniert den Rechner aus um Benutzerdaten, Screenshots, Tastatureingaben, Festplatteninhalte etc. zu erhalten
\end{description}
Die verschiedenen Arten von Trojanern werden oftmals kombiniert um so mächtige Werkzeuge zu erschaffen. 
Im Folgenden werden die drei bekanntesten Trojaner vorgestellt, welche außerdem als Grundlage für viele weiteren Trojaner dienen. 

Der \textbf{Zbot-Trojaner} ist ein Trojaner der \textit{Zeus}-Familie. Er war einer der ersten
Trojaner die als Framework mit einem Konfigurations-Tool zur Individualisierung verkauft wurde.
Ziel des Zbot-Trojaners ist es, möglichst viele Informationen wie Zugangsadaten des Benutzers zu E-Mail-Konten, sozialen Netzwerken oder Online-Banking zu stehlen,
um im Anschluss durch Erpressung vom Anwender Geld zu erlangen.
Die Informationen werden beispielsweise durch Protokollierung der Tastatureingaben gewonnen. 
Durch die Verwendung von \textit{stealth techniques} ist der Trojaner von Antiviren-Software sehr schwer zu erkennen.
Typischerweise versteckt er sich im Speicher indem der Originalspeicher kopiert wird und Anfragen von anderen Programmen
an diese weitergeleitet werden.
So kann der Trojaner unbemerkt den Speicher verändern.

Eine Abwandlung eines Zeus-Trojaners ist der \textbf{Reveton-Trojaner}.
Dieser setzt den klassischen Ransomware Ansatz um, mit dem Ziel den Zugriff auf das System zu blockieren und Lösegeld für die Freigabe zu fordern.
In Deutschland ist er bekannter unter dem Namen \enquote{BKA-Trojaner}.
Der Reveton-Trojaner verwendet offizielle Logos um gegenüber der Opfer seriös zu wirken.

Der \textbf{Vundo-Trojaner} verfolgt das Ziel Werbung für gefährliche bzw. gefälschte Antivirus-
Programme anzuzeigen und so Drive-By-Downloads zu starten. Der Trojaner installiert
meist eine Kombination aus 'Browser Helper Objects' und 'DLLs'. Um unerkannt zu bleiben
deaktiviert er außerdem Firewalls, Antiviren-Programme, Windows Updates, etc. 
\cite{BEKTRO}